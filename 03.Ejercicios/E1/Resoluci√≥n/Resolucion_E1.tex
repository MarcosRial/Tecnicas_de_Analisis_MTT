\documentclass[11pt,a4paper]{article}

\usepackage[utf8]{inputenc}
\usepackage[spanish]{babel}
\usepackage{amsmath}
\usepackage{amsfonts}
\usepackage{amssymb}
\usepackage{graphicx,geometry}
\author{Marcos Rial Docampo}
\title{Técnicas de Análisis Cuantitativas y Cualitativas\\Resolución del ejercicio de evaluación 1}
\date{\small{\today}}

\begin{document}
\pagestyle{empty}
\maketitle
\thispagestyle{empty}

En el ejercicio se presentan dos tipos de resultados de forma gráfica. Los primeros resultados son los referentes a la información extraída del resultado de las elecciones municipales de Madrid del 24 de mayo de 2015 en la que se muestra en forma de gráfico el porcentaje de votos recibidos por las diferentes formaciones políticas. Los segundos resultados son extraídos de una encuesta de estimación de voto en el municipio de Madrid hecha entre los días 4 y 10 de septiembre de 2015.

El primer fallo en lo referente a la presentación de los datos es la de la omisión de incluir en los datos de la encuesta los intervalos de confianza. En el primer caso no sería correcto incluir el intervalo de confianza puesto que se trata de datos objetivos extraídos de unos resultados electorales donde el tamaño muestral es el total de la población con derecho a voto del municipio de Madrid. No se trata de una encuesta como en el segundo caso, donde el tamaño muestral es de 502 personas consultadas.
\end{document}